\documentclass{article}
\usepackage{amsmath}
\usepackage{graphicx}
\usepackage{hyperref}
\usepackage{longtable}

\title{Parallel and Vectorized Matrix Multiplication}
\author{Your Name}
\date{\today}

\begin{document}

\maketitle

\section{Introduction}

In this assignment, we investigate parallel computing techniques, including multi-threading and libraries such as OpenMP, for matrix multiplication. Additionally, we compare the performance of vectorized approaches, utilizing SIMD instructions, with the basic matrix multiplication algorithm.

\section{Requirements}
\begin{itemize}
    \item Implement a parallel version of matrix multiplication.
    \item Test with large matrices and analyze the performance gain from vectorization and parallelization.
    \item \textbf{Optional}: Implement a vectorized version of matrix multiplication using SIMD instructions.
    \item \textbf{Optional}: Compare both approaches with the basic matrix multiplication algorithm.
\end{itemize}

\section{Metrics}
We focus on the following metrics to evaluate the performance of the algorithms:
\begin{itemize}
    \item \textbf{Speedup} compared to the basic algorithm.
    \item \textbf{Efficiency} of parallel execution (e.g., speedup per thread).
    \item \textbf{Resource usage}, including the number of cores used and memory consumption.
\end{itemize}

\section{Benchmark Results}

We conducted benchmarks on matrices of various sizes: 256x256, 512x512, 1024x1024, and 2048x2048. The results include execution time, memory allocation, and garbage collection metrics.

\subsection{Matrix 2048x2048}
\textbf{Execution Time (ms/op)}:
\begin{itemize}
    \item Mean: 31,691.818 ms
    \item Minimum: 31,273.157 ms
    \item Maximum: 32,117.444 ms
    \item Standard Deviation: 364.444 ms
    \item Confidence Interval (99.9\%): [30,288.475 ms, 33,095.161 ms]
\end{itemize}

\textbf{Memory Allocation (GC Allocations)}:
\begin{itemize}
    \item Average Allocation Speed: 1.102 MB/sec
    \item Minimum: 1.087 MB/sec
    \item Maximum: 1.116 MB/sec
    \item Standard Deviation: 12.342 MB/sec
    \item Confidence Interval (99.9\%): [1.054 MB/sec, 1.149 MB/sec]
    \item Memory Allocated per Operation: ~37.208 GB/op
\end{itemize}

\textbf{GC Churn (Generational Garbage Collection)}:
\begin{itemize}
    \item \textbf{G1 Eden Space:}
    \begin{itemize}
        \item Allocation Speed: 1.106 MB/sec
        \item Memory Allocated per Operation: ~37.356 GB/op
    \end{itemize}
    \item \textbf{G1 Old Gen:}
    \begin{itemize}
        \item Allocation Speed: 11.537 MB/sec
        \item Memory Allocated per Operation: ~0.388 GB/op
    \end{itemize}
    \item \textbf{G1 Survivor Space:}
    \begin{itemize}
        \item Allocation Speed: 0.716 MB/sec
        \item Memory Allocated per Operation: ~0.024 GB/op
    \end{itemize}
\end{itemize}

\textbf{Observations}:
\begin{itemize}
    \item The benchmark confirms that matrix multiplication with large matrices is expensive in terms of both computation time and memory.
    \item The average execution time for a 2048x2048 matrix exceeds 31 seconds.
    \item The average memory allocation is substantial, with over 37 GB of memory used per iteration.
    \item The garbage collector (GC) showed high activity, particularly in the Eden space, with churn speeds exceeding 1 GB/sec on average.
\end{itemize}

\subsection{Matrix 1024x1024}
\textbf{Execution Time (ms/op)}:
\begin{itemize}
    \item Average Time per Iteration: 3461.436 ± 426.822 ms/op
    \item Minimum: 3399.856 ms/op
    \item Maximum: 3659.112 ms/op
    \item Standard Deviation: 110.844 ms
    \item Confidence Interval (99.9\%): [3034.614 ms, 3888.258 ms]
\end{itemize}

\textbf{Memory Allocation (GC Alloc Rate)}:
\begin{itemize}
    \item Average: 1228.518 ± 137.931 MB/sec
    \item Minimum: 1164.694 MB/sec
    \item Maximum: 1249.016 MB/sec
\end{itemize}

\textbf{GC Churn in Eden Space}:
\begin{itemize}
    \item Average: 1221.808 ± 105.634 MB/sec
    \item Interval: [1116.173, 1327.442] MB/sec
\end{itemize}

\textbf{GC Time and Count}:
\begin{itemize}
    \item Total GC Count: 66 counts
    \item Average per Iteration: 13.2 counts
    \item Total GC Time: 1911 ms
    \item Average per Iteration: 382.2 ms
\end{itemize}

\subsection{Matrix 512x512}
\textbf{Execution Time (ms/op)}:
\begin{itemize}
    \item Average: 401.315 ± 56.457 ms/op
    \item Minimum: 387.109 ms
    \item Maximum: 425.982 ms
\end{itemize}

\textbf{Memory Allocation (GC Alloc Rate)}:
\begin{itemize}
    \item Average: 1325.449 ± 180.801 MB/sec
    \item Memory Allocated per Operation: ~585 MB
\end{itemize}

\textbf{GC Churn}:
\begin{itemize}
    \item G1 Eden Space: 1337.834 ± 154.668 MB/sec
    \item G1 Survivor Space: 6.297 ± 1.789 MB/sec
\end{itemize}

\textbf{GC Count and Time}:
\begin{itemize}
    \item Total GC Count: 102 counts
    \item GC Time: 861 ms
\end{itemize}

\subsection{Matrix 256x256}
\textbf{Execution Time (ms/op)}:
\begin{itemize}
    \item Average: 47.909 ± 4.592 ms/op
    \item Confidence Interval (99.9\%): [43.317 ms, 52.501 ms]
\end{itemize}

\textbf{Memory Allocation (GC Alloc Rate)}:
\begin{itemize}
    \item Average: 1413.002 MB/sec
    \item GC Alloc Rate Norm: 74.622 MB/op
\end{itemize}

\textbf{GC Churn (G1 Eden Space)}:
\begin{itemize}
    \item Average: 1418.136 MB/sec
\end{itemize}

\textbf{GC Count and Time}:
\begin{itemize}
    \item Total GC Count: 170 counts
    \item GC Time: 594 ms
\end{itemize}

\section{Thread State Profiling}
The thread states were observed during the execution:
\begin{itemize}
    \item \textbf{RUNNABLE}: The majority of the time (up to 78.3\% for large matrices), with the method \texttt{MatrixMultiplication.lambda$multiply$2} responsible for most of the execution time.
    \item \textbf{TIMED\_WAITING} and \textbf{WAITING}: Around 10\%-21\% of the time, indicating synchronization or idle periods.
    \item \textbf{BLOCKED}: A small portion of time (up to 2.9\%) due to concurrency issues such as accessing shared resources.
\end{itemize}

\section{Conclusion}

This benchmarking exercise demonstrated that matrix multiplication with large matrices is computationally expensive. The parallel and vectorized implementations showed significant improvements in execution speed, but memory usage remains a limiting factor. The performance gains from parallelization and vectorization were evident, though further optimizations could be explored for even larger matrices or more specialized hardware.

\end{document}
