\documentclass{article} \usepackage{hyperref}

\begin{document}

\section{Procedures for Running the Algorithms}

\subsection{1. Python} To run the Python code, you need to have Python installed and set up an Anaconda environment with the necessary libraries.

\textbf{Steps:}

\textbf{Install Python:}

Download and install \textbf{Anaconda} following the instructions for your operating system from \url{https://www.anaconda.com/products/distribution}.
\textbf{Create a Conda Environment:}

Open the terminal (or Anaconda Prompt) and create a new environment with the following command: \begin{verbatim} conda create --name matrix-env python=3.9 \end{verbatim}
Activate the created environment: \begin{verbatim} conda activate matrix-env \end{verbatim}
\textbf{Install Necessary Libraries:}

Install the required libraries (e.g., NumPy for matrix manipulation): \begin{verbatim} pip install numpy \end{verbatim}
\textbf{Run the Python Code:}

Navigate to the folder containing your Python script (e.g., \texttt{matrix.py}) and execute the following command: \begin{verbatim} python matrix.py \end{verbatim}
The results will be displayed in the console.
\subsection{2. C} To compile and run the C code, you need to have the necessary libraries installed.

\textbf{Steps:}

\textbf{Install a C Compiler:}

Ensure you have a C compiler such as \textbf{MinGW} for Windows. During installation, select options to include the binary files in your PATH.
\textbf{Compile the C Code:}

Open the command prompt (cmd) and navigate to the folder containing your C file (e.g., \texttt{matrix.c}). Compile the code with the following command: \begin{verbatim} gcc matrix.c -o matrix.exe \end{verbatim}
\textbf{Run the C Code:}

After compilation, execute the generated file: \begin{verbatim} matrix.exe \end{verbatim}
\textbf{Compare Results:}

Once the code is running, the results of the matrix multiplication will be displayed in the terminal.
\subsection{3. Java} Running the Java code is different as it uses JMH (Java Microbenchmark Harness) for benchmarking. You need to install Maven and configure the project.

\textbf{Steps:}

\textbf{Install Maven:}

Download Maven from \url{https://maven.apache.org/download.cgi} and follow the instructions for setting it up on your system.
\textbf{Create a Maven Folder:}

Open the terminal and create a new folder for your project: \begin{verbatim} mvn archetype
-DgroupId=com.example -DartifactId=matrix-benchmark -DarchetypeArtifactId=maven-archetype-quickstart -DinteractiveMode=false \end{verbatim}
This command creates a predefined directory structure for a Maven project.
\textbf{Navigate to the Project Folder:}

Change into the created folder: \begin{verbatim} cd matrix-benchmark \end{verbatim}
\textbf{Project Structure:}

The folder contains subdirectories like \texttt{src/main/java} for source code and \texttt{src/test/java} for tests. You can place your benchmarking code inside \texttt{src/main/java/com/example}.
\textbf{Use Maven Commands:}

Run the following commands to prepare the project: \begin{verbatim} mvn clean install package \end{verbatim}
\textbf{Explanation of commands:}
\texttt{clean}: removes previous build files.
\texttt{install}: compiles the project and installs the package in the local repository.
\texttt{package}: creates an executable JAR file for the project.
\textbf{Run the JAR:}

To execute the benchmark, use the command: \begin{verbatim} java -jar target/matrix-benchmark-1.0-SNAPSHOT.jar \end{verbatim}
\textbf{Warmup and Interactions:}

Warmup executions prepare the JVM and reduce noise in the results. Interactions are the actual benchmarking tests during which the performance of the algorithm is measured.
\end{document}

